% Last Modification:
% @author AUTHOR_NAME
% @date TODAY_DATE

\chapter{Introduction}
\label{chap:introduction}

\section{Overview}
\msricrash is a simple system dedicated to any person who wants to inform of a
car crash crisis situation in order to allow for crisis handling.
At anytime and anywhere, anyone can be the witness or victim of a car crash and might be
in a situation allowing for alerting this crisis. 
The \msricrash system has for objectives to support crisis declaration and secure administration and crisis handling by the \msricrash professional users. 

\section{Purpose and recipients of the document}
This document is an analysis document complying with the \msrmessir
methodology~\cite{messirbook}. Its intent is to provide an example of a precise
specification of the functional properties of the \msricrash system. \\

The recipients of this document are:
\begin{itemize}
\item the \msricrash system's buyer company (ABC): this document is used as a
contractual document jointly with any other document considered as useful (as
requirement elicitation document, \ldots) in order to have a higher degree of
precision in requirement description. It is also used as a basis document
for the \msricrash system validation using specification based testing.
\item the \msricrash system development company (ADC) is expected to use this document as
the basis for development (mainly design, implementation, maintenance). It is also
 used for verification and validation using test plans defined using the
analysis models described in this document and according to the \msrmessir methodology.
 
\end{itemize} 

 
\section{Application Domain}
The \msricrash system belongs to the Crisis Management Systems Domain. It is a
system dedicated to crisis professional and non professional end users. It has
to be considered as an autonomous and external service for the society. 
It is not an institutional system certified and guaranteed by any governmental
entity and thus, must be used with caution.

 
\section{Definitions, acronyms and abbreviations}

N.A.

\section{Document structure} 
The document structure is designed to be coherent with the
\msrmessirmeth~\cite{messirbook}. Section \ref{chap:general_description} provides a general
description of the system purpose, its users, its environment and some general
non functional requirements. A more detailed description of the non functional
requirements, if any, are provided in section~\ref{chap:additional_constraints}.
The \glspl{system operation} triggered by events sent by the external
\glspl{actor}  belonging to the environment are described in Section \ref{chap:lu.uni.lassy.excalibur.examples.icrash-EM}.
The \msricrash concepts used to represent the any persistent or transient
information is given in Section \ref{chap:lu.uni.lassy.excalibur.examples.icrash-CM}. The precise specification of the system operations in term of system's state changes, events sent together with the constraints on the allowed sequences of system operations are described in Section
\ref{chap:lu.uni.lassy.excalibur.examples.icrash-OM}.
